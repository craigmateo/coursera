\documentclass[13.5pt]{article}
\usepackage[margin=1in]{geometry}
\usepackage{fancyhdr}
\pagestyle{fancy}
\usepackage{amssymb}
\usepackage[usenames, dvipsnames]{color}
\usepackage[T1]{fontenc}
\usepackage{biblatex}

\lhead{KEITH DEVLIN: Introduction to Mathematical Thinking}
\chead{}
\rhead{ASSIGNMENT 1 (for Lecture 1)}

\begin{document}

\begin{enumerate}
\item{Find two unambiguous (but natural sounding) sentences equivalent to the sentence \textit{The man saw the woman with a telescope}, the first where the man has the telescope, the second where the woman
has the telescope}\\

\begin{enumerate} 
  \textcolor{blue}{i. The man with the telescope saw the woman.}\\
  \textcolor{blue}{ii. The man saw the woman who was holding a telescope.} 
\end{enumerate}

\item{For each of the three ambiguous newspaper headlines I stated in the lecture, rewrite it in a way
that avoids the amusing second meaning, while retaining the brevity of a typical headline:
}
\begin{enumerate}
\item{Sisters reunited after ten years in checkout line at Safeway.}\\\\
\textcolor{blue}{After ten years apart sisters meet in checkout line at Safety.}\\
\item{Large hole appears in High Street. City authorities are looking into it.}\\\\
\textcolor{blue}{Large hole appears in High Street. City authorities are investigating.}\\
\item{Mayor says bus passengers should be belted.}\\\\
\textcolor{blue}{Mayor says bus passengers should buckle-up.}\\
\end{enumerate}

\item{The following notice was posted on the wall of a hospital emergency room:}\\\\
\centerline{\textsc{No head injury is too trivial to ignore.}}\\\\
{Reformulate to avoid the unintended second reading. (The context for this sentence is so strong that many people have difficulty seeing there is an alternative meaning.)}\\\\
\textcolor{blue}{\centerline{\textsc{There is not a head injury which is too trivial to ignore.}}}

\item{You often see the following notice posted in elevators:}\\\\ \centerline{\textsc{In case of fire, do not use elevator.}}\\\\
{This one always amuses me. Comment on the two meanings and reformulate to avoid the unintended
second reading. (Again, given the context for this notice, the ambiguity is not problematic.)}\\\\
\textcolor{blue}{\centerline{\textsc{If there's a fire, do not use elevator.}}}

\item{Official documents often contain one or more pages that are empty apart from one sentence at the bottom:}\\\\
\centerline{\textit{This page intentionally left blank.}}\\\\
{Does the sentence make a true statement? What is the purpose of making such a statement?
What reformulation of the sentence would avoid any logical problems about truth? (Once again,
the context means that in practice everyone understands the intended meaning and there is no
problem. But the formulation of a similar sentence in mathematics at the start of the twentieth
century destroyed one prominent mathematician's seminal work and led to a major revolution in
an entire branch of mathematics.)
}\\\\
\textcolor{blue}{The sentence is not literally true since the page is not actually blank. The blank page may be there to separate the main content from material in the first pages. The purpose of the statement is likely to alert the reader that it is not a mistake. A reformulated sentence could refer more specifically to the intent. For example, a sentence like \textit{This page is a separator} would avoid the logical problems.}
\definecolor{gray}{gray}{0.6}
\item{Find (and provide citations for) three examples of published sentences whose literal meaning is (clearly) not what the writer intended. (This is much easier than you might think. Ambiguity is
very common.)}\\\\
\textcolor{blue}{i. "New insights into the inner clock of the fruit fly."}
\textcolor{gray}{\textit{source: New insights into the inner clock of the fruit fly. https://phys.org/news/2018-05-insights-clock-fruit.html, accessed: 08.30.2018}}\\\\
\textcolor{blue}{ii. "Unlike the congressional committees digging into the Russia affair, Mueller has the authority to lay criminal charges."}
\textcolor{gray}{\textit{source: Republican in Russia probe 'even more convinced' FBI didn't spy on Trump campaign. http://www.cbc.ca/news/world/gowdy-trump-sessions-russia-1.4683700, accessed: 08.30.2018}}\\\\
\textcolor{blue}{ii. "She is positioned right beside last year's stable horse."}
\textcolor{gray}{\textit{source: Indiana Grand Racing Club Announces Purchase Of Deputy Storm Filly. https://www.paulickreport.com/news, accessed: 08.30.2018}}

\item{Comment on the sentence "The temperature is hot today." You hear people say things like this all the time, and everyone understands what is meant. But using language in this sloppy way in mathematics would be disastrous.}\\\\
\textcolor{blue}{The temperature refers to an amount and, literally speaking, the adjective "hot" cannot apply to the number itself but to the sensation. }

\item{How would you show that not every number of the form \(N=(p_1 \cdot p_2 \cdot p_3 ... \cdot p_n) + 1\)  is prime, where \(p_1,p_2,p_3...p_n,...\)  is the list of all prime numbers?}\\\\ \textcolor{blue}{A counter example. One could also write a proof based on the fact that \(p_2=2\) and the product of primes must therefore be an even number. }

\end{enumerate}

JUST FOR FUN

\begin{enumerate}

\item{Provide a context and a sentence within that context, where the word \textit{and} occurs five times in succession, with no other word between those five occurrences. (You are allowed to use punctuation.)}\\\\
\textcolor{blue}{Say someone is telling you something and they stop before any conclusion. You want them to give more information so you keep repeating: and... and..., and..., and..., and....}

\item{ Provide a context and a sentence within that context, where the words \textit{and, or, and, or, and} occur in that order, with no other word between them. (Again, you can use punctuation.)}\\\\
\textcolor{blue}{Perhaps only in a silly assignment where you are giving instructions (like in the question above).}
\end{enumerate}
\end{document}